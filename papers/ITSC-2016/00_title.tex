\title{The \textsc{TransP-0} framework for integrated transportation and power system design}

\author{
	\IEEEauthorblockN{Dominik Ascher}
	\IEEEauthorblockA{
		Fakult\"at f\"ur Informatik\\
		Technische Universit\"at M\"unchen\\
		85748 Garching bei M\"unchen, Germany\\
		Email: \href{mailto:ascher@in.tum.de}{ascher@in.tum.de}
	}
	\and
	\IEEEauthorblockN{Georg Hackenberg}
	\IEEEauthorblockA{
		Fakult\"at f\"ur Informatik\\
		Technische Universit\"at M\"unchen\\
		85748 Garching bei M\"unchen, Germany\\
		Email: \href{mailto:hackenbe@in.tum.de}{hackenbe@in.tum.de}
	}
}

\maketitle

\begin{abstract}
	High penetration of electric vehicles (EV) and renewable energy sources (RES) will require fundamental changes to prevalent transportation and power systems. Intermittent and decentralized loads within the power system caused by RES and EV and the propagation of new transportation paradigms such as the transportation electrification and mobility-on-demand services will impose critical, closely interrelated changes on these systems.
	To guarantee sustainability and minimize negative environmental impacts, closer integration between transportation and power systems is necessary and integrated planing, operation and control strategies for these systems have to be established. In this paper, we present TRANSP-0, a system design abstraction for rapid and iterative formulation and evaluation of design options within integrated transportation and power systems. Firstly, we present the TRANSP-0 design space in terms of the static parameters for intelligent transportation and energy subsystem design. Secondly, we present the dynamic aspects of the subsystem design expressed by according optimal control problem formulation. Thirdly, we discuss the requirements to control strategies expressed by objectives and constraints of the underlying optimal control problems. Finally, we conclude with an outlook on the future scope of the proposed system design abstraction.
	
\end{abstract}
