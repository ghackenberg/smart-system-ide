\section{States, actions, and transition function}
\label{dynamics}

While the previous section was concerned only with the static parameters of integrated transportation and energy system design, this section focuses on dynamic aspects instead. In effect, each system design defines an optimal control problem (or dynamic programming problem)~\cite{Bertsekas1995} over the transportation and energy subsystem dynamics. In the following, we describe the respective state space in Section~\ref{states}, the action space in Section~\ref{actions}, and the transition function in Section~\ref{transitions}. Note that the states, actions, and transitions do not have to be defined by the transportation and power system engineers. Rather, the definitions are the same for all system design expressed in the \textsc{TransP-0} abstraction.

\subsection{States}
\label{states}

The overall system states $S_t \in \mathbb{S}$ with time point $t \in \mathbb{N}$ of the optimal control problem are modeled as a four-tuple $(VS_t, ESS_t, CSS_t, RS_t)$, where
\begin{itemize}
	\item $VS_t$ represents the states of the \textit{vehicles} introduced in Section~\ref{vehicles},
	\item $ESS_t$ represents the states of the \textit{energy storages} introduced in Section~\ref{energy_storages},
	\item $CSS_t$ represents the states of the \textit{charging stations} introduced in Section~\ref{charging_stations}, and
	\item $RS_t$ represents the states of the \textit{regions} introduced in Section~\ref{regions}.
\end{itemize}
Note that we do not associate a state with the infrastructure of the transportation subsystem (i.e.\ we assume the infrastructure to be constant). In the following, we describe the vehicle states in Section~\ref{states_vehicles}, the energy storage states in Section~\ref{states_storages}, the charging station station states in Section~\ref{states_stations}, and the region states in Section~\ref{states_regions}.

\subsubsection{Vehicles}
\label{states_vehicles}

The vehicle states $VS_t$ of the system state $S_t$ are modeled as a tuple $(VP_t, VSOC_t)$, where
\begin{itemize}
	\item $VP_t: VL \rightarrow RSP$ represents a mapping from vehicle labels to road segment \textit{positions} (see Section~\ref{segments}) and
	\item $VSOC_t: VL \rightarrow \mathbb{R}_0^+$ represents a mapping from vehicle labels to their \textit{state of charge} (i.e. the amount of currently stored energy).
\end{itemize}
\todo{Wrap-up.}

\subsubsection{Energy storages}
\label{states_storages}

The energy storage states $ESS_t$ of the system state $S_t$ are modeled as a one-tuple $(ESOC_t)$, where
\begin{itemize}
	\item $ESOC_t: ESL \rightarrow \mathbb{R}_0^+$ represents a mapping from energy storage labels to their current \textit{state of charge}. 
\end{itemize}
\todo{Wrap-up.}

\subsubsection{Charging stations}
\label{states_stations}

The charging stations states $CSS_t$ of the system state $S_t$ are modeled as a one-tuple $(CSB_t)$, where
\begin{itemize}
	\item $CSB_t: CSL \rightarrow \mathbb{R}$ represents a mapping from charging station labels to the current charging station \textit{balance} (i.e. the amount of energy sent or received from a connected vehicle).
\end{itemize}
\todo{Wrap-up.}

\subsubsection{Regions}
\label{states_regions}

The region states $R_t$ of the system state $S_t$ are modeled as a one-tuple $(RB_t)$, where
\begin{itemize}
	\item $RB_t: RL \rightarrow \mathbb{R}$ represents a mapping from region labels to the current region \textit{balance} (i.e. the aggregated loads of connected energy subsystem regions and components).
\end{itemize}
\todo{Wrap-up.}

\subsection{Actions}
\label{actions}

The actions $A_t \in \mathbb{A}$ with time point $t \in \mathbb{N}$ of the optimal control problem are modeled as a tuple $(VA_t, ESA_t)$, where
\begin{itemize}
	\item $VA_t$ represents the actions of the \textit{vehicles} introduced in Section~\ref{vehicles} and
	\item $ESA_t$ represents the actions of the \textit{energy storages} introduced in Section~\ref{energy_storages}.
\end{itemize}
Note that the vehicles and the energy storages are the only system components comprising actions. The states of the other components is influenced directly or indirectly by these actions. In the following, we describe the vehicle actions in Section~\ref{actions_vehicles} before explaining the energy storage actions in Section~\ref{actions_storages}.

\subsubsection{Vehicles}
\label{actions_vehicles}

The vehicle actions $VA_t$ of the system action $A_t$ are modeled as a three-tuple $(VR_t, VS_t, VB_t)$, where
\begin{itemize}
	\item $VR_t: VL \rightarrow (\mathbb{N} \rightarrow RSL)$ represents a mapping from vehicle labels to their respective \textit{route}, i.e.\ a sequence of connected road segments with $\forall vl \in VL, n \in \mathbb{N}:$
	\[
		RST(VR_t(vl)(n)) = RSS(VR_t(vl)(n + 1))
	\]
	starting at the road segment position of the previous vehicle states with $\forall vl \in VL$ and $VP_t(vl) = (rsl, d):$
	\[
		VR_t(vl)(0) = rsl \textrm{,}
	\]
	\item $VS_t: VL \rightarrow \mathbb{R}_0^+$ represents a mapping from vehicle labels to the current vehicle \textit{speed} (i.e.\ the velocity of the vehicle along the road segments), and
	\item $VB_t: VL \rightarrow \mathbb{R}$ represents a mapping from vehicle labels to vehicle \textit{balances} (i.e.\ the amount of energy sent to or received from a charging station).
\end{itemize}
\todo{Wrap-up.}

\subsubsection{Energy storages}
\label{actions_storages}

The energy storage actions $ESA_t$ of the system action $A_t$ are modeled as a one-tuple $(ESB_t)$, where
\begin{itemize}
	\item $ESB_t: ESL \rightarrow \mathbb{R}$ represents a mapping from energy storage labels to energy storage \textit{balances} (i.e.\ the amount of energy sent to or received from the parent region).
\end{itemize}
\todo{Wrap-up.}

\subsection{Transition function}
\label{transitions}

The transition function is a mapping
$
	T: \mathbb{S} \times \mathbb{A} \rightarrow \mathbb{S}
$ describing the transition from a state, given an action to a new state.
Stepping forward in time, the transition function is then defined as a function
\[
	T(S_t, A_t) = S_{t+1} \mathrm{,}
\]
where $S_{t+1}$ describes the state following $S_t$ after choosing action $A_t$. Accordingly, the transition function $T$ is then

The individual vehicle state
\begin{itemize}
	\item $VS_{t+1} = f(VS_t, VA_t)$ represents ...,
	\item $ESS_{t+1} = g(ESS_t, ESA_t)$ represents ...,
	\item $CSS_{t+1} = h(VS_{t+1}, VA_t)$ represents ..., and
	\item $RS_{t+1} = i(CSS_{t+1}, ESA_t)$ represents ....
\end{itemize}
\todo{Outline.}

\subsubsection{Vehicles}
\label{transitions_vehicles}

\todo{Introduction.}
\begin{itemize}
	\item $VP_{t+1} = f_1(VP_t, VR_t, VS_t)$ represent ... and
	\item $VSOC_{t+1} = f_2(VP_t, VSOC_t, VR_t, VS_t, VB_t)$ represents ....
\end{itemize}
\todo{Wrap-up.}

\subsubsection{Energy storages}
\label{transitions_storages}

\todo{Formalize!}

\subsubsection{Charging stations}
\label{transitions_stations}

\todo{Formalize!}

\subsubsection{Regions}
\label{transitions_regions}

\todo{Formalize!}