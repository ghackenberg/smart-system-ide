\documentclass[conference]{IEEEtran}

\usepackage{amsmath}
\usepackage{amssymb}
\usepackage{graphicx}
\usepackage{color}
\usepackage[colorlinks,allcolors=blue]{hyperref}

\newcommand{\todo}[1]{\textcolor{red}{/* #1 */}}

\begin{document}
	
	\title{A model for multi-objective optimal multi-modal passenger transport and energy flow control}
	
	\author{
		\IEEEauthorblockN{Dominik Ascher}
		\IEEEauthorblockA{
			Fakult\"at f\"ur Informatik\\
			Technische Universit\"at M\"unchen\\
			85748 Garching bei M\"unchen, Germany\\
			Email: \href{mailto:ascher@in.tum.de}{ascher@in.tum.de}
		}
		\and
		\IEEEauthorblockN{Georg Hackenberg}
		\IEEEauthorblockA{
			Fakult\"at f\"ur Informatik\\
			Technische Universit\"at M\"unchen\\
			85748 Garching bei M\"unchen, Germany\\
			Email: \href{mailto:hackenbe@in.tum.de}{hackenbe@in.tum.de}
		}
		\and
		\IEEEauthorblockN{...}
		\IEEEauthorblockA{
			...
		}
	}

	\maketitle
	
	\begin{abstract}
		\todo{Write abstract.}
	\end{abstract}
	
	\section{Introduction}
	
	Guaranteeing sustainability and minimizing negative environmental impacts are crucial challenges for future power and transportation systems. Widespread adoption of electric vehicles (EVs) as well as high penetration of renewable energy sources (RES) causes substantial changes within these systems. Here, within the power systems, intermittent power outputs require elaborate load balancing strategies while increasing adoption of autonomous electric vehicles requires addressing changing mobility demands in the transportation system. To comprehensively address constraints, objectives and design alternatives of both transportation and power systems, sustainable and integrated planning, operation and control strategies have to be established within integrated transportation and power systems.
	
	According planning, operation and control strategies are frequently  addressed within the concept of vehicle-to-grid (V2G). V2G describes a concept, where energy is released from EV batteries to the power system in times of increased power demand.  By facilitating interaction between power system and EVs, V2G possesses the potential to significantly reduce the amount of excess renewable energy produced within the power system \cite{richardson2013electric}. More generally, significant advantages as well as disadvantages can be argued for widespread V2G adoption. Here, both environmental and economic benefits have been shown in the past \cite{faria2012sustainability, richardson2013electric, mwasilu2014electric}. Furthermore, key benefits of V2G include reduction of emissions, increased efficiency as well as stability and reliability of the power system \cite{yilmaz2013review}.
	
	On the contrary, Mwasilu et al.~\cite{mwasilu2014electric} argue that for V2G adoption in a smart grid context central technological issues have to be addressed first such as coping with communication delays, establishing routing protocols and cyber security. Adding to that, the authors argue that frequent charging and discharging cycles causing rapid battery degradation for EV batteries as well as low penetration of electric vehicles with V2G functionality hinder more widespread V2G adoption.
	
	Nevertheless, previous work on plug-in electric vehicles (PEVs) has shown their ability to contribute to balancing the fluctuation of intermittent renewable energy sources \cite{dallinger2012grid}. Currently, V2G represents a supplementary strategy to address situations in the power system with increased power demand. Currently, available intermittent renewable energy sources have to be supplemented with conventional energy sources as well as energy storages within the power system. To achieve a continuous balance between energy supply and demands within power systems, prevalent control schemes rely on automatic control schemes which are supervised by human operators. Here, according control schemes are subject to challenges from increasing levels of fluctuating RES \cite{heussen2012unified}.
	
	Given these current and future challenges and the prospected impact of RES and EVs on such systems, effective integrated control strategies to handle future power and mobility demands still have a long way to go and are subject to ongoing research.
	
	\todo{Multi-modal passenger transport?}
	
	\todo{Multi-objective optimal control?}
	
	\subsubsection*{Outline}
	
	The remainder of the article is structured as follows: Section~\ref{related_work} summarizes related work in the field. Section~\ref{proposed_work} summarizes the contributions of this article. Section~\ref{proposed_model} describes the proposed modeling technique. Section~\ref{discussion} discusses the proposed technique with respect to various quality criteria. Finally, Section~\ref{conclusion} draws a conclusion from our current state of work.
	
	\section{Related work}
	\label{related_work}
	
	In the following we first review related approaches in Section~\ref{approaches} before deriving remaining problems in Section~\ref{problems}.
	
	\subsection{Approaches}
	\label{approaches}
		
	Previous studies have shown that, given a high penetration of EVs, uncontrolled charging can impose increased peak loads within the distribution network \cite{lopes2009identifying}. Intelligent scheduling methods are widely discussed as key approaches to integrate electric vehicles into the power grid \cite{yang2015computational}. According approaches often focus on minimizing single or multiple objectives within given power system models, while restricting valid behavior in terms of a set of constraints. For instance, typical objectives are minimizing cost (or maximizing welfare), power losses, emissions, power deviations or optimizing battery performance of EVs within power systems \cite{yang2015computational}. Here, the power systems are consisted of a number of electric devices such as conventional or renewable energy sources, energy consumers as well as electric infrastructure.
	
	In this context, Andreotti et al.~\cite{andreotti2012review} compare single-objective optimization methods within a smart grid context under the presence of EVs to evaluate model effectiveness in terms of operational limits and used objective functions. In this context, to sufficiently address technical and economic objectives for PEVs, the authors argue higher suitability of multi-objective optimization methods. 
	
	Zakariazadeh et al.~\cite{zakariazadeh2014multi} propose a multi-objective scheduling method for electric vehicles within a smart distribution network addressing economic and environmental objectives as well as technical constraints. Here, the approach manages to reduce operational costs and emissions, achieving Pareto-optimal solutions.
	
	To achieve optimal charging decisions for EVs, Ota et al.~\cite{ota2012autonomous} propose a decentralized V2G control scheme to address the intermittency of RES energy production using electric vehicles. However, the authors focus on the effects of an according charging control scheme within an isolated power system only.
	
	Here, approaches often restrict the impact of electric vehicles to decisions on charging or discharging their batteries at charging stations. However, in subsequence, individual EV objectives describing routing preferences such as shortest traveling time or energy-efficiency for EVs cannot be sufficiently taken into account. Instead, emphasis is put on the power system side, while the transportation system including traffic participants isn't represented microscopically. 
	
	Another highly relevant direction for efficiently integrating electric vehicles into the power grid and reduce negative impacts is are approaches utilizing Vehicle Routing Problems (VRPs). Methods for vehicle routing typically focus on optimizing route selection for single or multiple traffic participants towards single or multiple objectives and a given set of constraints. Addressing objectives of energy-efficiency in terms of routing problems, Eco-Routing approaches target energy-efficient route selection. In contrast, Eco-Driving approaches target energy-efficient intermediate driving behavior ~\cite{ericsson2006optimizing}.
	
	Felipe et al.~\cite{felipe2014heuristic} propose multiple heuristics for routing electric vehicles which consider different partial recharge strategies and recharge technologies while traveling along routes. However, approaches does not take the effects of recharging within the power system into account for general cost evaluation. 
	
	Integrating both scheduling and routing approaches for EV, Barco et al.~\cite{barco2013optimal} present an approach for minimizing operation cost for battery electric vehicle (BEV) fleets. While the authors propose a methodology which focuses on optimal routing and scheduling of charge for EV fleets, the approach does not consider effects on the power system when making routing and charging decisions in EVs.
	
	\todo{Multi-modal passenger transport?}
	
	\todo{Multi-objective optimal control?}
	
	\todo{Subsubsections?}
	
	\subsection{Problems}
	\label{problems}
	
	In summary, we found that current approaches do not sufficiently address the objectives and constraints of both transportation and power systems to holistically estimate the effects of future power and transportation system scenarios. While approaches for scheduling EVs heavily address the effects of EVs within the power system, they neglect their effects on the transportation system. In contrast, routing approaches for EVs heavily address the effects of single or multiple EVs within the transportation system, in which routes are optimized, but neglect a detailed representation of the power system and it's underlying objectives.
	
	\todo{Multi-modal passenger transport?}
	
	\todo{Multi-objective optimal control?}
	
	\section{Proposed work}
	\label{proposed_work}
	
	In the following we first describe the theoretical backgrounds of our work in Section~\ref{backgrounds} before summarizing the contributions of this article in Section~\ref{contributions}
	
	\subsection{Backgrounds}
	\label{backgrounds}
	
	In \cite{Hackenberg2012} we presented a model of the electric power system suitable for large-scale computation. The model divides the power system into regions and subregions. In each time step for each region the power balance is calculated as the sum of all subregion power balances.
	
	Then, in \cite{ascher2014early} we presented a model that captures the mobility demands of individual vehicles within transportation systems. For this, the technique employs a representation which formulates multi-objective traffic flows as optimal control problems. Furthermore, the transportation infrastructure is represented as directed graph, where the edges and the distances traveled on edges represent the positions of electric vehicles.
	
	Finally, in \cite{ascher2015integrated} we presented a component-based model which allows one to express static interaction (e.g. between vehicle and controller) as well as dynamic interaction between components (e.g. vehicle and charging station). Here, the presented modeling approach allows one to microscopically model power systems based on individual electric devices and transportation systems based on individual cars in terms of components. We then proposed a integrated transportation and power system model, which allows to capture the respective demands of both transportation and power systems.
	
	\subsection{Contributions}
	\label{contributions}
	
	In this work, we extend our previous work and present a more extensive and detailed formulation of our model for integrated transportation and power systems. 
	
	Within the transportation system mobility demands are expressed by passengers, who impose (1)~position preferences including origin and destination of travel as well as (2)~time preferences, which include departure and arrival times. Different modes of transportation (pedestrian, electric vehicle, public transportation) can be used by passengers to satisfy the mobility demands. 
	
	\todo{Multi-modal passenger transport.}
	
	Within the power system, power demands are expressed by electric devices in terms of electric power loads (or energy flows) within specific times and durations. For being able to satisfy power demands, the power system can employ different energy sources (renewable, conventional), which require resources (coal, gas) and cause emissions resulting from the employed resources.
	
	\todo{Formalization (not done before).}
	
	\todo{Computation (i.e.\ some example)?}
	
	\section{Proposed model}
	\label{proposed_model}
	
	In the following we describe our model for integrated multi-objective optimal multi-modal passenger transport and energy flow control. Our model is illustrated in Figure~\ref{illustration}. The transportation system is modeled using a directed graph, where the nodes represent intersections and the edges represent transport segments. Furthermore, the edges are colored with respect to the mode of transport (e.g.\ pedestrian, EV, public transport). In contrast, the power system is modeled using a tree, where the nodes represent regions and the edges represent a containment relationship.
	
	\begin{figure}
		\centering
		\includegraphics[width=\columnwidth]{gfx/model.pdf}
		\caption{Illustration of integrated multi-modal passenger transport and energy flow modeling. The directed graph represents the transportation system. The nested ellipses and circles represent the power system.}
		\label{illustration}
	\end{figure}
	
	Section~\ref{modalities} introduces are model of transport modalities. Section~\ref{transport} describes our model of the transport infrastructure. Section~\ref{passengers} explains our passenger model. Section~\ref{vehicles} highlights our vehicle model. Section~\ref{collisions} defines our collision criteria. Section~\ref{power} summarizes our model of the power infrastructure. Section~\ref{charging_stations} describes our model of charging stations. Section~\ref{power_batteries} explaing our model of power batteries. Section~\ref{solar_panels} defines our model of solar panels (i.e.\ power generators). Finally, Section~\ref{static_loads} summarizes our model of static loads (i.e.\ non-controllable loads).
	
	\todo{How to structure this section? E.g.\ static vs.\ dynamic aspects? Transportation vs.\ power system? Contraints, objectives, ...? States, actions, ...?}
	
	\subsection{Transport modalities}
	\label{modalities}
	
	Transport modality identifiers $TM$ (finite set)
	\[
		 \exists n \in \mathbb{N} : |TM| < n
	\]
	\todo{Pedestrian, bicycle, EV, bus, metro, short distance train, long distance train, airplane, ...}
	
	\subsection{Transport infrastructures}
	\label{transport}
	
	Road intersection identifiers $RI$ (finite set)
	\[
		\exists n \in \mathbb{N} : |RI| < n
	\]
	Road segments $RS$ (connectivity)
	\[
		RS \subseteq RI \times RI
	\]
	Road segment modality $RSM$ \todo{Use!}
	\[
		RSC : RS \rightarrow TM
	\]
	Road segment lanes $RSL$ (capacity)
	\[
		RSL : RS \rightarrow \mathbb{N}
	\]
	Road segment distance $RSD : RS \rightarrow \mathbb{R}^+$ (zero-distance self-segments)
	\[
		(ri, ri) \in RS \Rightarrow  RSD(ri, ri) = 0
	\]
	Road segment coordinates $RSC \subseteq RS \times \mathbb{R}^+$ (respect segment distance)
	\[
		RSC = \{(\cdot, rd) \in RS \times \mathbb{R} \mid 0 \leq rd \leq RSD(s) \}
	\]
	World coordinates $WC$ (geography, latitude, longitude, elevation) \todo{Do we really need that in the presentation? Can we simplify the model using world coordinates of road intersections only?}
	\[
		WC : RSC \rightarrow \mathbb{R}^3
	\]
	Transport infrastructure $TI$ (topology)
	\[
		TI = (RI, RS, RSL, RSD, WC)
	\]
	
	\subsection{Passengers}
	\label{passengers}
	
	Passenger identifiers $P$ (finite set)
	\[
	\exists n \in \mathbb{N} : |P| < n
	\]
	Passenger mobility demand $PMD$ (origin position, destination position, departure time, arrival time)
	\[
		PMD \subseteq RSC \times RSC \times \mathbb{R}_0^+ \times \mathbb{R}_0^+
	\]
	Passenger state $PS$ (road segment coordinates, passenger mobility demand)
	\[
		PS : P \rightarrow RSC \times PMD
	\]
	\todo{What about passenger weight and size? Should we use those parameters? Do we need to integrate the bin-packing problem then? Is that necessary at the current stage?}
	
	\subsection{Vehicles (i.e.\ transport modes)}
	\label{vehicles}
	
	Vehicles $V$ (finite set)
	\[
		\exists n \in \mathbb{N} : |V| < n
	\]
	Vehicle modality $VM$ \todo{Use!}
	\[
		VM: V \rightarrow \{...\}
	\]
	Vehicle length $VL$ (for collision detection/avoidance)
	\[
		VL : V \rightarrow \mathbb{R}^+
	\]
	Vehicle (energy) capacity $VC$ (battery size)
	\[
		VC : V \rightarrow \mathbb{R}^+
	\]
	Vehicle weight $VW$
	\[
		VW : V \rightarrow \mathbb{R}^+
	\]
	Vehicle passenger capacity $VP$
	\[
		VP : V \rightarrow \mathbb{R}^+
	\]
	Vehicle energy-efficiency $VE$ 
		\[
		VE : V \rightarrow \mathbb{R}^+
	\]
	Vehicle charge-rate $VR$ 
		\[
		VR : V \rightarrow \mathbb{R}^+
	\]
%	Vehicle goods capacity $VG$
%	\[
%		VG : V \rightarrow \mathbb{R}^+
%	\]
	Vehicle state $VS : V \rightarrow RSC \times \mathbb{R}_0^+ \times \mathbb{R}_0^+$ (road segment position, charge state) \todo{Integrate passenger assignment. Not all passengers have to be assigned. Unassigned passengers must use the pedestrian modality.}
	\[
		VS(v) \in \{ (\cdot, cs, ps) \in RSC \times \mathbb{R}_0^+ \times \mathbb{R}_0^+ \mid
	\]
	\[
		(0 \leq cs \leq VC(v)) \wedge (0 \leq ps \leq VP(v)) \}
	\]
	Vehicles states $\mathbb{VS}$ \todo{Require that passengers are not assigned multiple times! Note that we could distinguish between properties that are enforced (e.g.\ unique passenger assignment), and properties that need to be solved (e.g.\ collision-free passenger/vehicle operation).}
	\[
		\mathbb{VS} = \{VS : V \rightarrow RSC \times \mathbb{R}_0^+ \mid VS \text{ is vehicle state}\}
	\]
	\todo{Do we really need all these vehicle parameters? Maybe we can reduce the model to a few parameters only? What do we loose then?}
	
	\subsection{Collisions}
	\label{collisions}
	
	Overlapping vehicle pairs $OVP : \mathbb{VS} \rightarrow V \times V$
	\[
		OVP(VS) = \{(v_1, v_2) \in V \times V \mid
	\]
	\[
		((rs_1,rd_1),\cdot) \in VS(v_1), ((rs_2,rd_2),\cdot) \in VS(v_2) :
	\]
	\[
		rs_1 = rs_2 \wedge (|rd_1 - rd_2| < VL(v_1) / 2
	\]
	\[
		\vee
	\]
	\[
		|rd_1 - rd_2| < VL(v_2) / 2)\}
	\]
	Overlapping vehicle sets $OVS : \mathbb{VS} \times V \rightarrow \mathcal{P}(V)$
	\[
		OVS(VS,v) = \{v' \in V \mid (v, v') \in OVP(VS)\}
	\]
	Collision property $CV : \mathbb{VS} \rightarrow \mathbb{B}$
	\[
		CV(VS) \Leftrightarrow \exists v \in V :
	\]
	\[
		|OVS(VS, v)| > RSL(rs) \text{ with } ((rs,\cdot),\cdot) = VS(v)
	\]
	
	\subsection{Power infrastructures}
	\label{power}
	
	Voltage net identifiers $VN$ \todo{Rename to region!}
	\[
		\exists n \in \mathbb{N} : |VN| < n
	\]
	Voltage net parent $VNP$ \todo{Require acyclic parent relationship with distinct root.}
	\[
		VNP : VN \rightarrow VN
	\]
	Voltage net capacity $VNC$ \todo{The maximum energy flow through this region.}
	\[
		VNC : VN \rightarrow \mathbb{R}^+
	\]
	Voltage net state $VNS$ \todo{The current energy balance in each region. Calculate the balance from all connected regions and devices!}
	\[
		VNS : VN \rightarrow \mathbb{R}
	\]
	\todo{Maybe we should introduce electric components here already. Then, electric components can be regions or end-points. Regions are the net identifiers introduced above. End-points are the concepts introduced in the following.}
	
	\subsection{Charging stations}
	\label{charging_stations}
		
	Charging station identifiers $CS$
	\[
		\exists n \in \mathbb{N} : |CS| < n
	\]
	Charging station position $CSP : CS \rightarrow RS$
	\[
		CSP(cs) = (ri_1, ri_2) \Rightarrow ri_1 = ri_2
	\]
	\todo{Map charging station to regions.}
		
	\subsection{Power batteries}
	\label{power_batteries}
	
	Power battery identifiers $PB$	
	\[
		\exists n \in \mathbb{N} : |PB| < n
	\]
	Power battery capacity $PBC$
	\[
		PBC : PB \rightarrow \mathbb{R}^+
	\]
	Power battery state $PBS : PB \rightarrow \mathbb{R}_0^+$ (current capacity)
	\[
		PBS(pb) \in \{ cc \in \mathbb{R}_0^+ \mid 0 \leq cc \leq PBC(pb) \}
	\]
	\todo{Use energy storages instead of power batteries (because they are more general).}
	\\
	\todo{Map power batteries to regions.}
	
	\subsection{Solar panels}
	\label{solar_panels}
	
	Solar panel identifiers $SP$
	\[
		\exists n \in \mathbb{N} : |SP| < n
	\]
	Solar panel power scale $SPS$ 
	\[
		SPS : SP \rightarrow \mathbb{R}^+
	\]
	Solar panel power $SPP$ 
	\[
		SPP : SP \times \mathbb{R}^+ \rightarrow \mathbb{R}^+
	\]
	\todo{Use power generators instead of solar panels (because they are more general). Then we have to model the emissions (CO2, noise, etc.) also! Solar panels do not cause any emissions (only during production and maintenance).}
	\\
	\todo{Map power generators to regions.}
	
	\subsection{Static (i.e.\ non-controllable) loads}
	\label{static_loads}
	
	Static load identifiers $SL$ \todo{Represent non-controllable energy production and consumption. For example, activities such as cooking or watching television are non-controllable in our model. In general, the loads can be selected according to the use case.}
	\[
		\exists n \in \mathbb{N} : |SL| < n
	\]
	Static load profile $SLP$ (active/reactive power)
	\[
		SLP : SL \times \mathbb{R}_0^+ \rightarrow \mathbb{R} \times \mathbb{R}
	\]
	\todo{Map static loads to regions.}
	
	\subsection{Contraints}
	
	\todo{What are the constraints in our model? Vehicle collision, vehicle overloading, electric net capacity overloading, ...}
	
	\subsection{Objectives}
	
	\todo{What are the objectives in our model? Derivation from departure time, derivation from arrival time, derivation from departure road segment, derivation from arrival road segment, emissions, ...}
	
	\section{Discussion}
	\label{discussion}
	
	\todo{Write discussion. What does the model capture? How well does the model capture this information? What does the model not capture? When do we need the information that is not captured? When don't we need the information that is not captured?}
	
	\section{Conclusion}
	\label{conclusion}
	
	\todo{Write conclusion. How far are we in modeling the problem? When can we use the model for real-world scenarios? How far are we in solving the problem?}
	
	\bibliographystyle{bst/IEEEtran}
	\bibliography{references}
	
\end{document}