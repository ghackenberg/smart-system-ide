\section{Related Work (0.5 pages)}

In recent time, approaches dealing with the optimal charging scheduling for electric vehicles have seen widespread attention in research. 
% approaches diverse in scope and method to handle the problem.
Sundstrom et al. \cite{sundstrom2010planning} consider the capacity of the electricity grid in the charging of electric vehicles, thereby avoiding overloading in the grid.
An approach by Schlote et al. \cite{schlote2012balanced} implements load-balancing strategies for charging demand at charging points into the routing decisions of electric vehicles, resulting in reduced travel times and congestion. 
% multiple objectives of shortest travelling time as well as demand alleviation correlate in case
%However, rapidly changing objectives 
%are considered through exploring alternative options in terms of charging stations to drive to, based on current suitability. 
%However, the specific adaption to different objectives, parameters for estimating differing situations is not evaluated. 
Furthermore, attempts to integrate intelligent charging behavior of electric vehicles into smart grids have been made. Alonso et al. \cite{alonso2014optimal} examine optimal charging scheduling for electric vehicles in smart grids, achieving optimal behavior for charging of EVs within a representative low-voltage net topology. The objective is oriented towards obtaining daily optimal scheduling of EV demand on transformer substations.

However, common among these approaches is that demand alleviation within the power grid represents the main objective. Therefore, differing or additional objectives as well as constraints of the traffic system, it's individual EVs and the power grid can not be represented. Furthermore, differing configurations in regard to the objectives cannot be explored and analyzed. 

%A further drawback is that energy-efficient routing or intermediate driving behavior is neglected, while these concepts have been shown to increase sustainability in the past. 
%Also, explicit objectives of individual electric vehicles and an detailed grid infrastructure with devices, rather than profiles is not considered in detail. 

Soares et al. \cite{soares2012electric} propose an electric vehicle scenario simulator, which enables the definition of electric vehicle scenarios in the context of smart grids and distribution networks.
However, it's scope for scenario simulation is limited to the definition of electric vehicles scenarios and relies on external tools for specific analysis and determining impact on the power grid.


%Focus has been placed on among balancing charging demand among electric vehicles and charging stations within a given traffic infrastructure


%\subsection{Eco-Routing/Eco-Driving Approaches}

%Eco-routing/-driving approaches consider energy-efficient driving behavior on a given route of individual traffic participants. 

%They don’t consider interactions with and objectives of charging stations and the electrical infrastructure. 
%Furthermore, Eco-Routing/-Driving approaches do not employ global planning of the individual traffic participant’s routes and interactions, therefore coordination between multiple vehicles remains an issue, which is %only partially alleviated by incorporating real-time traffic information. Because of this, especially Eco-Routing approaches are susceptible to congestions and do not address rapidly changing objectives.

%Approaches for balancing charging demand mainly focus on the interactions between EVs and charging stations within a given traffic infrastructure independent of electricity infrastructure interactions and it’s %reactiveness to changing objectives. 

In summary, all considered approaches have in common that they aim to address the objectives of energy-efficient system behavior.
The issue remains that none of the approaches allow to rapidly explore and evaluate holistic transportation scenarios due to limited scope and intended objectives. Therefore, interfacing components of infrastructures are only partially or not considered, leading to insufficient observations about the feasibility of future transportation scenarios.
