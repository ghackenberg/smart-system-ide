\section{Conclusion}
\label{section:conclusion}

In this paper, we proposed a parametric approach to holistic transportation system and electric network scenario modeling. We proposed and demonstrated an extension to a suitable modeling technique adapted to the ITS domain relying on partial system models. 
In our approach, we proposed an integrated system, being able to model power systems with their electric devices and infrastructures as well as transportation systems with individual cars. Our approach allows for varying the parameters within transportation system and electric network scenarios. We evaluated several incremental traffic scenarios, whose parameters have been varied in terms of renewable and smart energy penetration and in terms of balance of the objectives of transportation and power systems. 
%Findings demonstrated the feasibility of our approach for holistic transportation system and electric network scenario modeling. 
This showed possibilities for evaluating the objectives of future systems as well as system composition. Future work includes refinement of model accuracy through vehicle model improvements, utilization of additional objectives, modeling of additional systems and integration of OpenStreetMap to model representative traffic infrastructures. Additionally, validation of results obtained with our approach is a major next step. Finally, to be able to handle larger model problem complexities more easily, we currently work on distributed behavior estimation algorithms.

%Integration in SUMO. 
%Independently validation of behavior estimation results could be shown by comparison to SUMO implementation.
%Further extension of our approach could include the modeling and parametrization of additional systems such as weather
%Balancing local voltage net loads 