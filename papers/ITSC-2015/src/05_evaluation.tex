\begin{table*}[b]
	\centering
	\renewcommand{\arraystretch}{1.3}
	\begin{tabularx}{\textwidth}{|Y|Y|Y|Y|}
		\hline
		
		\textbf{Scenario} 1 & \textbf{Scenario 2} & \textbf{Scenario 3} & \textbf{Scenario 4} \\
		
		\hline
		
		\includegraphics[width=0.23\textwidth, trim=0 0 0 -3]{../gfx/data/E1_003.png} &
		\includegraphics[width=0.23\textwidth, trim=0 0 0 -3]{../gfx/data/E2_003.png} &
		\includegraphics[width=0.23\textwidth, trim=0 0 0 -3]{../gfx/data/E3_003.png} &
		\includegraphics[width=0.23\textwidth, trim=0 0 0 -3]{../gfx/data/E4_003.png} \\
		
		\hline
		
		\includegraphics[width=0.23\textwidth, trim=0 0 0 -3]{../gfx/data/E1_001.png} &
		\includegraphics[width=0.23\textwidth, trim=0 0 0 -3]{../gfx/data/E2_001.png} &
		\includegraphics[width=0.23\textwidth, trim=0 0 0 -3]{../gfx/data/E3_001.png} &
		\includegraphics[width=0.23\textwidth, trim=0 0 0 -3]{../gfx/data/E4_001.png} \\
		
		\hline
		
		\includegraphics[width=0.23\textwidth, trim=0 0 0 -3]{../gfx/data/E1_002.png} &
		\includegraphics[width=0.23\textwidth, trim=0 0 0 -3]{../gfx/data/E2_002.png} &
		\includegraphics[width=0.23\textwidth, trim=0 0 0 -3]{../gfx/data/E3_002.png} &
		\includegraphics[width=0.23\textwidth, trim=0 0 0 -3]{../gfx/data/E4_002.png} \\
		
		\hline
		
		\begin{tabulary}{4cm}{L|L}
			\textit{Computation time} & 229s \\
			\textit{Memory usage} & 6.03GB \\
		\end{tabulary}
		&
		\begin{tabulary}{4cm}{L|L}
			\textit{Computation time} & 250s \\
			\textit{Memory usage} & 5.27GB \\
		\end{tabulary}
		&
		\begin{tabulary}{4cm}{L|L}
			\textit{Computation time} & 218s \\
			\textit{Memory usage} & 6.41GB \\
		\end{tabulary}
		&
		\begin{tabulary}{4cm}{L|L}
			\textit{Computation time} & 227s \\
			\textit{Memory usage} & 6.75GB \\
		\end{tabulary}
		\\
		
		\hline
	\end{tabularx}
	\caption{Traffic flow graph, power chart, statistics, computation time, memory usage, and lines of code for each scenario.}
	\label{figure:examples}
\end{table*}

\section{Demonstration}
\label{section:evaluation}

To demonstrate the presented approach, we provide four examples with specific scenario configurations. Generally, all examples are simulated with a time resolution of 60 seconds per simulation step and 30 simulation steps in total, aggregating to a total duration of 30 minutes. The examples evaluate the effects of varying the weights $w\_TS$ and $w\_PS$ between the transportation system cost and the power system cost as well as the state of expansion of the power system.

To model the stage of expansion of the power system, between Scenarios 1-2 and Scenarios 3-4 the parameters of power system $PS$ are varied in terms of the number of charging stations $CS\_Number$ as well as in terms of solar panel capacities $Power\_Scale$ and power battery capacities $SOC\_Max$. Moreover, all scenarios include ten static load components, five solar panel components and five power battery components represented through their reference configurations $SL_{A}$, $SP_{A}$ and $PB_{A}$ or their respective stages of expansion $SL_{B}$, $SP_{B}$ and $PB_{B}$. From Scenarios 1-2 to Scenarios 3-4, solar panel capacities are increased twofold, while battery capacities are increased fourfold. Furthermore, in Scenarios 1-2 16 charging stations are employed, while Scenarios 3-4 utilize 56 charging stations. In all scenarios charging stations with reference configuration $CS_{A}$ are used. In terms of low/medium voltage nets and reference configurations, four low-voltage nets $LV_{A}$ and one medium-voltage net $MV_{A}$ are employed in all scenarios. 

In terms of the transportation system $TS$, the examples feature a total number of 440 cars $C$. The cars are distributed equally among the reference configurations $C_{A}$, $C_{B}$ and $C_{C}$. Their reference configurations differ in the initial state of charge $SOC$ with 33\%, 66\% and 100\% of the maximum state of charge $SOC\_Max$ respectively. In terms of positioning on the traffic network $TN$, the origin positions $Origin$ of cars are drawn randomly from all edges $E$ present within the traffic network. Instead, the destination positions $Destination$ are drawn randomly from a set of most outward edges of the traffic network. Finally, for origins we prefer the lower left sector of $TN$, while for destinations we prefer the upper right sector of $TN$.

The results of control optimization are shown in Tab.~\ref{figure:examples}. For each scenario, a traffic flow graph and power chart is provided as well as general statistics and performance characteristics.

\subsubsection*{Scenario 1}

In Scenario~1, solar panel and power battery capacities available within the power system are low, while static profiles within the power system have high impact in terms of intermittent power loads. Furthermore, only a low number of charging stations is available. In benefit of achieving the objectives of the transportation system, higher weight is assigned to the costs incurred by the transportation system instead of the power system. In terms of traffic flows, behavior estimation results show high frequency of routes representing shortest paths to destinations. Net balance is prone to high and sudden fluctuations in load as little equalization of net balances is performed by cars charging or discharging at charging stations.

\subsubsection*{Scenario 2}

In contrast to Scenario~1, in benefit of achieving the objectives of the power system, higher weight is assigned to the costs incurred by the power system instead of the transportation system. In terms of traffic flows, behavior estimation results show higher frequency of edges around and on charging stations. Furthermore, compared to Scenario~1, cars utilize shortest path routes to destinations less frequently, which results in marginally higher total distance traveled by cars. Also in contrast to Scenario~1, net balance is more equalized, traceable to cars intermittently discharging at a low number of charging stations.

\subsubsection*{Scenario 3}

In Scenario~3 a stage of expansion of the power system is modeled. For this, higher solar panel and power battery capacities as well as a higher number of charging stations are employed. As electric devices get smarter, the profiles of static loads are more evened out as less intermittent power load peaks occur. In benefit of achieving the objectives of the transportation system, higher weight is assigned to the costs incurred by the transportation system instead of the power system. Similar to Scenario~1, when considering traffic flows, behavior estimation results show high frequency of routes utilizing shortest paths to destinations. Overall distance traveled by cars is decreased, while final distance to destination is increased. Compared to Scenario~2, a higher level of equalization of net balance can be observed, due to a higher number of charging stations enabling cars to charge/discharge and higher capacities of solar panels and power batteries.

\subsubsection*{Scenario 4}

In Scenario 4 higher weight is assigned to the costs incurred by the power system instead of the transportation system. Similar to traffic flows observed in Scenario~2, behavior estimation results show that cars utilize shortest path routes less frequently compared to Scenarios~1 and~3, which results in marginally higher distance traveled and final distance to destination. However, compared to Scenario~2, frequency is distributed more evenly across the traffic network, resulting in higher frequency of edges around and on charging stations compared to Scenario~3. Furthermore, due to cars charging and discharging at a high number of charging stations, in contrast to Scenario~3, net balances are equalized to a higher degree.