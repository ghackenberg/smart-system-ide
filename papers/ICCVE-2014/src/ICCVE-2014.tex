\documentclass[conference]{../cls/IEEEtran}

\usepackage{graphicx}

\begin{document}

\title{Early Estimation of Multi-Objective Traffic Behavior}

\author{
	\IEEEauthorblockN{Dominik Ascher}
	\IEEEauthorblockA{
		Chair IV: Software \& Systems Engineering\\
		Technische Universit\"at M\"unchen\\
		Boltzmannstr.\ 3, 85748 Garching, Germany\\
		Email: ds.ascher@gmail.com
	}
	\and
	\IEEEauthorblockN{Georg Hackenberg}
	\IEEEauthorblockA{
		Chair IV: Software \& Systems Engineering\\
		Technische Universit\"at M\"unchen\\
		Boltzmannstr.\ 3, 85748 Garching, Germany\\
		Email: hackenbe@in.tum.de
	}
}

\maketitle

\begin{abstract}
Intelligent Transportation Systems (ITS) have come a long way targeting recent problems of increasing emissions and growing vehicle numbers. Current approaches address a variety of objectives including congestion management, collision avoidance, energy-efficiency and emission reduction. However, respective solutions typically are designed for and tailored to a fixed subset of objectives. Consequently, the effects of changing objectives and priorities cannot be assessed easily. To overcome this situation we present a lightweight approach to estimating traffic behavior early in the systems engineering process using discrete-time and continuous-state system models. We demonstrate the feasibility of the framework using a basic traffic scenario before concluding with an initial discussion and outlook.
\end{abstract}

\begin{IEEEkeywords}
Feasibility study, intelligent transportation systems.
\end{IEEEkeywords}

\section{Motivation}

In recent decades, efforts to engineer ITS have come a long way, targeting contemporary and future problems of increasing emissions and growing vehicle numbers. Among the approaches two main directions can be distinguished: (1) Urban Traffic Control (UTC) focusing on multiple traffic participants and their interaction~\cite{Chen2010,Dresner2008} and (2) Eco-Routing focusing on a single traffic participant and his route~\cite{Ericsson2006, Barth2007,Boriboonsomsin2012}. In particular, UTC typically targets congestion management and collision avoidance~\cite{Chen2010}. Technically, the approaches rely on agent based techniques to describe human driving behavior with respect to traffic supervision and control~\cite{Chen2010}. More advanced approaches even consider fully autonomous vehicles thereby proposing an alternative to human-oriented traffic supervision and control~\cite{Dresner2008}. In contrast, Eco-Routing typically targets energy-efficiency and emission reduction~\cite{Ericsson2006,Barth2007}. Technically, the approaches rely on optimization techniques to describe route variables, constraints and objectives~\cite{Ericsson2006,Barth2007}. As frequent use of energy-efficient routes might cause congestion, advanced approaches finally incorporate historical and real-time traffic information thereby taking first steps towards bridging the gab between Eco-Routing and UTC~\cite{Boriboonsomsin2012}. Common among all previous approaches are objectives like congestion management, collision avoidance, energy-efficiency and emission reduction.

However, while the presented approaches offer tailored solutions for a fixed set of objectives, they are not designed to estimate the effect of changing objectives and priorities. In particular, it is important to estimate this effect in early phases of systems engineering (such as requirements discovery or feasibility study) to uncover potential opportunities and threads~\cite{Whitten2005}. To overcome this situation in the following we present a lightweight model-based approach.

\section{A Lightweight Model-Based Approach}

\begin{figure}[b]
	\centering
	\includegraphics{../gfx/framework.pdf}
	\caption{Overview of the model-based approach to traffic behavior estimation.}
	\label{fig:framework}
\end{figure}
An overview of our model-based approach to traffic behavior estimation is shown in Figure~\ref{fig:framework}. Technically, we base our work on emergent property estimation techniques described in~\cite{Hackenberg2012}. Consequently, we use a discrete-time and continuous-state system model. We rely on discrete-time models to reduce the reachable state space during behavior estimation. However, we work with continuous-state models because quantities like velocity or distance can be described more intuitively. Further, we employ a generic model architecture separating between context, constraint, objective, software and equivalence model. The non-deterministic software model describes the optimizable control behavior (not the control logic), while the context model describes the controlled physical state. Based on the physical state the constraint and objective models define the operational limits and costs. Finally, the equivalence model describes how the physical states can be clustered during optimization. For control behavior estimation the framework finally integrates stochastic approximation techniques~\cite{Pereira1991} to achieve robustness against arbitrary cost functions/constraints.

\section{Its Application to Traffic Systems}

Our objective is to study the feasibility and effects of multiple objectives on
urban traffic systems. For this, we evaluate different weight configurations
towards multiple objectives to study driving behavior of vehicles. Firstly, in
the model section, we describe the model architecture and its underlying
behavior. Secondly, in the analysis section, we then apply our model to a
specific case.

\subsection{Model}

Simulation steps in our approach are measured in discrete time
intervals, which can be adjusted accordingly in the model.
The traffic system in our model is represented through a directed graph
consisted of nodes and edges between them. Nodes are defined through their absolute position:
latitude, longitude and elevation. Edges are defined through respective source
and target node and assigned number of traffic lanes.

In the model architecture, individual vehicles are defined through according
components: \textit{The behavior of the vehicle control component is essentially
defined by speed and edge selection. Edge selection is based on the available choices on the current position of the observed vehicle. If a vehicle has
completely crossed an edge, it is able to select a new edge to travel to. Speed
selection depicts an average value for the time segment and depends on a
continuous average value range. The vehicle context component's behavior is defined by
relative position on the edge, state of charge and vehicle energy consumption. A
simplified approach towards energy consumption estimation is employed: Energy consumption within a given time
segment is approximately determined via the current edge's altitude difference 
and average speed. Vehicle speed has a quadratic relationship to observed
energy consumption. In the equivalence component, the vehicles current speed is defined
for clustering during state space exploration. The vehicle objective component,
determines the operational goals and vehicle driving behavior in the traffic
system. Respective objectives are formulated as cost functions, which can
differently weighted in respect to each other. Cost function A measures the
elapsed time since the start of the simulation until reaching the target. Cost function B determines the vehicles current
energy consumption in relation to the vehicles maximum possible energy consumption.
Furthermore, in the traffic system objective component, different priorities of
vehicles in the traffic system can be introduced through weighting of their individual costs.
Finally, in the vehicle constraint component, a constraint tests, whether the
current state of charge lies within minimum and maximum limits. Moreover, in the
traffic system constraint component, a collision constraint tests for every possible
pair of vehicles whether the sum of their currently selected relative position
on the edge and the vehicles length overlap and whether there are free lanes
available on the current position based on edge capacity.}

\subsection{Analysis}

For illustration of our approach, we assume a simple case of multiple vehicles
within a traffic network: Each vehicle has the same origin and the same
destination. There are two feasible, energetically and distance different
routes for vehicles to drive from the origin to the destination. Route A
represents a more lengthy, flat route, while route B depicts a shorter route
with differing heights. To evaluate among different weighting strategies towards
the objectives of vehicles, we illustrate two weight configurations.
According simulation results for vehicle edge selection and aggregated energy
consumption over time are depicted in Figure 2. 

\begin{figure}[t!]
	\includegraphics[width=\columnwidth]{../gfx/placeholder.pdf}
	\caption{Top: Comparison of the different control objectives in Case A. Bottom:
	Comparison of control objectives in Case B.}
	\label{figure:results}
\end{figure}


\section{Conclusion and Outlook}

We presented first steps towards an estimation of a balancing
approach of multiple control objectives for traffic behavior.
Preliminary behavior estimation results of driving behavior have been obtained and provided insight 
in possible trade-off considerations between multiple control objectives.
Further research has to show the feasibility of more specific objectives and
assigned weighting for individual vehicles. 

\bibliographystyle{../bst/IEEEtran}
\bibliography{ICCVE-2014}

\end{document}
