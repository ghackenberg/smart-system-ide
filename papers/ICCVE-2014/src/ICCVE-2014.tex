\documentclass[conference]{../cls/IEEEtran}

\usepackage{graphicx}

\begin{document}

\title{Behavior Estimation on Energy-Efficient Urban Traffic Control to Explore
Operational Feasibility}

\author{
	\IEEEauthorblockN{Dominik Ascher}
	\IEEEauthorblockA{
		Chair IV: Software \& Systems Engineering\\
		Technische Universit\"at M\"unchen\\
		Boltzmannstr.\ 3, 85748 Garching, Germany\\
		Email: ds.ascher@gmail.com
	}
	\and
	\IEEEauthorblockN{Georg Hackenberg}
	\IEEEauthorblockA{
		Chair IV: Software \& Systems Engineering\\
		Technische Universit\"at M\"unchen\\
		Boltzmannstr.\ 3, 85748 Garching, Germany\\
		Email: hackenbe@in.tum.de
	}
}

\maketitle

\begin{abstract}
Many application ideas arise from increasing computational and communicational capabilities of today's automotive vehicles.
Early studies are needed to show the feasibility and potential of these ideas to foster future research and development.
\end{abstract}

\begin{IEEEkeywords}
Feasibility study, urban traffic control, eco-routing.
\end{IEEEkeywords}

\section{Motivation}

In recent decades, efforts to engineer Intelligent Transportation Systems (ITS)
have come a long way, targeting contemporary and future problems of increasing
emissions and global vehicle number growth.  On the one hand, elaborate traffic control
mechanisms have been established. Specifically, Urban Traffic Control
(UTC) approaches deal with the objectives of congestion and collision avoidance
between multiple traffic participants through elaborate traffic control systems ~\cite{Chen2010}.
More efficient, future-oriented traffic control approaches in the context of
intersection management ~\cite{Dresner2008} discuss autonomous vehicles interacting among
themselves using artificial intelligence, thereby erasing the paradigm of human drivers and conventional
human coordination systems like traffic lights.

On the other hand, routing techniques handling the objectives of energy-efficiency
and emission reduction have seen widespread recognition in
research as so-called Eco-Routing ~\cite{Ericsson2006, Barth2007}. As more
energy-efficient routes can also be subject to more congestion due to increased
usage by multiple traffic participants, ~\cite{Boriboonsomsin2012} propose an approach incorporating historical and real-time traffic information within an Eco-Routing routing technique, thus
extending the aforementioned objectives to multiple vehicles.
\begin{figure}[b!]
	\includegraphics[width=\columnwidth]{../gfx/overview-variant.pdf}
	\caption{Overview of the general systems engineering process from requirements discovery over feasibility study, system design and implementation to system operation adapted to cooperative eco-routing.}
	\label{figure:process}
\end{figure}
However, while the reviewed approaches offer feasible solutions
towards specific objectives, they do not enable to
balance and analyze multiple objectives and their resulting behavior in traffic
systems.

To overcome this situation, in this paper, we report on the first steps towards
an operational feasibility study on multiple objective control strategies in traffic systems. We shortly describe the
underlying behavior estimation framework before explaining the multiple objective control demonstrator as well as initial performance analysis results.

\section{Behavior Estimation Framework}

To illustrate the scope of our work we first give a short definition of feasibility studies: 
According to Whitten et al.~\cite{Whitten2005} in systems engineering feasibility studies follow the requirements discovery phase to uncover potential opportunities and threads.
Upon successful feasibility demonstration the system design, implementation and operation phases follow.

We translate this process to cooperative eco-routing as illustrated in Figure~\ref{figure:process}.
Given the requirements (i.e.\ energy-efficient and congestion-free routing) we develop a control spectrum specification and analyze it with respect to an approximate physical model.
Technically, we base our work on early emergent property estimation techniques described in~\cite{Hackenberg2012}.
Consequently, we use a discrete-time and continuous-state system model as depicted in Figure~\ref{figure:framework}.
We rely on discrete-time models to reduce the reachable state space during analysis.
However, we use continuous-state models because quantities like velocity or distance can be described more intuitively.
Moreover, we rely on a generic and reusable model architecture dividing the system into software, context, constraint, objective and equivalence components.
In particular, the partial software model describes the control spectrum, while the context model describes the physical state.
Based on the physical state the constraint and objective models define the operational limits and goals.
Finally, the equivalence model describes how the physical states can be clustered during state space exploration.
Behavior estimation uses stochastic optimization techniques similar to~\cite{Pereira1991} to approximate optimal system behavior.

\section{Energy-Efficient Urban Traffic Control}

Based on the aforementioned models and their related architecture, our approach
is consisted of individual vehicles according components and their individual
behavior. Simulation steps in our approach are measured in discrete time
intervals of 5 minutes. The behavior of the vehicle control component is
essentially defined by speed and edge selection. Edge selection is based on the available
choices on the current position of the observed vehicle. If a vehicle has
completely crossed an edge, it is able to select a new edge to travel to. Speed
selection depicts an average value for the time segment and depends on a
continuous average value range. The vehicle context component's behavior is defined by
relative position on the edge, charge state and vehicle energy consumption. A simplified approach towards energy
consumption estimation is employed: Energy consumption within a given time
segment is approximately determined via the current edge's altitude difference 
and average speed. Vehicle speed has a quadratic relationship to observed
energy consumption. In the equivalence component, the vehicles current speed is defined
for clustering during state space exploration. The vehicle objective component,
determines the operational goals and vehicle driving behavior in the traffic
system. Respective objectives are formulated as cost functions, which can
differently weighted in respect to each other. Cost function A measures the
elapsed time since the start of the simulation until reaching the target. Cost function B determines the vehicles current
energy consumption in relation to the vehicles maximum possible energy consumption.
Furthermore, in the traffic system objective component, different priorities of
vehicles in the traffic system can be introduced through weighting of their individual costs.
Finally, in the vehicle constraint component, a constraint tests, whether the
current charge state lies within minimum and maximum limits. Moreover, in the traffic
system constraint component, a collision constraint tests for every possible
pair of vehicles whether the sum of their currently selected relative position
on the edge and the vehicles length overlap and whether there are free lanes
available on the current position based on edge capacity.

For behavior estimation, we demonstrate a basic case of a
single vehicle within a traffic system with two energetically and
distance different routes. Both routes have the same starting points and target
destinations. While route A represents a longer, flat route, route B depicts a
shorter route with differing heights. According simulation results for speed selection and energy consumption over time are depicted in Figure 2 (Top). In a second case, we
compare how well our control approach copes with a multitude of vehicles and a larger traffic network with energetically different route profiles. In Figure 2 (bottom), the 
results to the evolution of speed selection and energy consumption for the
second case are depicted.

\begin{figure}[t!]
	\includegraphics[width=\columnwidth]{../gfx/placeholder.pdf}
	\caption{Top: Comparison of the different control objectives in Case A. Bottom:
	Comparison of control objectives in Case B.}
	\label{figure:results}
\end{figure}


\section{Conclusion and Outlook}

We presented first steps towards studying the feasibility of a balancing
approach of multiple control objectives for urban traffic systems.
Preliminary behavior estimation results of driving behavior have been obtained and provided insight 
in possible trade-off considerations between multiple control objectives.
Further research has to show the feasibility of the approach towards a larger
problem size, more specific objectives and their respective balancing.

\bibliographystyle{../bst/IEEEtran}
\bibliography{ICCVE-2014}

\end{document}
