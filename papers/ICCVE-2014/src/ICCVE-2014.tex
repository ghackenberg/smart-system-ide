\documentclass[conference]{../cls/IEEEtran}

\usepackage{graphicx}

\begin{document}

\title{Behavior Estimation Framework on Cooperative Eco-Routing to Explore Operational Feasibility}

\author{
	\IEEEauthorblockN{Dominik Ascher}
	\IEEEauthorblockA{
		Chair IV: Software \& Systems Engineering\\
		Technische Universit\"at M\"unchen\\
		Boltzmannstr.\ 3, 85748 Garching, Germany\\
		Email: ds.ascher@gmail.com
	}
	\and
	\IEEEauthorblockN{Georg Hackenberg}
	\IEEEauthorblockA{
		Chair IV: Software \& Systems Engineering\\
		Technische Universit\"at M\"unchen\\
		Boltzmannstr.\ 3, 85748 Garching, Germany\\
		Email: hackenbe@in.tum.de
	}
}

\maketitle

\begin{abstract}
Many application ideas arise from increasing computational and communicational capabilities of today's automotive vehicles.
Early studies are needed to show the feasability and potential of these ideas to foster future research and development.
\end{abstract}

\begin{IEEEkeywords}
Feasibility study, cooperative eco-routing.
\end{IEEEkeywords}

\section{Motivation}

State of the art: eco-routing (energy efficient, fast) + Urban traffic control (cooperative)

Problem: Cooperative eco-routing approaches are missing

To overcome this situation in this paper we report on first steps towards a holistic operational feasibility study of cooperative eco-routing.
In the following we describe shortly the underlying framework before explaining the cooperative eco-routing demonstrator and initial performance results.

\section{Behavior Estimation Framework}

To illustrate the scope of our work we first give a short definition of feasibility studies: 
According to Whitten et al.~\cite{Whitten2005} in systems engineering feasibility studies follow the requirements discovery phase to uncover potential opportunities and threads.
Upon successfull feasibility demonstration the system design, implementation and operation phases follow.

\begin{figure}[b]
	\centering
	\includegraphics{../gfx/process.pdf}
	\caption{Overview of the general systems engineering process from requirements discovery over feasibility study, system design and implementation to system operation adapted to cooperative eco-routing.}
	\label{figure:process}
\end{figure}
We translate this process to cooperative eco-routing as illustrated in Figure~\ref{figure:process}.
Given the requirements (i.e.\ energy-efficient and congestion-free routing) we develop a control spectrum specification and analyze it with respect to an approximate physical model.
Technically, we base our work on early emergent property estimation techniques described in~\cite{Hackenberg2012}.
Consequently, we use a discrete-time and continuous-state system model as depicted in Figure~\ref{figure:framework}.
We rely on discrete-time models to reduce the reachable state space during analysis.
However, we use continuous-state models because quantities like velocity or distance can be described more intuitively.
Moreover, we rely on a generic and reusable model architecture dividing the system into software, context, constraint, objective and equivalence components.
In particular, the partial software model describes the control spectrum, while the context model describes the physical state.
Based on the physical state the constraint and objective models define the operational limits and goals.
Finally, the equivalence model describes how the physical states can be clustered during state space exploration.
Behavior estimation uses stochastic optimization techniques similar to~\cite{Pereira1991} to approximate optimal system behavior.

\begin{figure}[b]
	\centering
	\includegraphics{../gfx/framework.pdf}
	\caption{Overview of the estimation framework including component-based system model (including non-deterministic software component), behavior estimator and system behavior.}
	\label{figure:framework}
\end{figure}

\section{Cooperative Eco-Routing}

Model architecture: Context (i.e.\ vehicle) component, constraint component, objective component, control component, equivalence component.

Component behavior: Speed selection, edge selection, edge-based collision detection, energy consumption, etc.

Behavior estimation: One basic case (single vehicle, two routes) and one more complex case (multiple vehicles).

\section{Conclusion and Outlook}

Great stuff! :)

\bibliographystyle{../bst/IEEEtran}
\bibliography{ICCVE-2014}

\end{document}
